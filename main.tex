\documentclass[fontsize=10.5bp]{jlreq}
\usepackage{master}

\begin{document}

\header{
    \centering
    \Title{修士論文中間試問会資料フォーマット}{Format of Proceedings for Preliminary Examination of Master's Thesis}
    \Authors{社会 一郎}{Ichiro Shakai}
    % \Authors{著者1姓 著者1名$^{1}$, 著者2姓 著者2名$^{2}$, 著者3姓 著者3名$^{1,2}$}
    % \Affiliations{$^{1}$所属1,$^{2}$所属2}
    % \Emails{authors1@e-mailaddress}
    \Abstruct{
        要旨を400字(英語の場合は250 words)以内で簡潔に記述すること。要旨は1段組、左右マージン25 mmとすること。○○○○○○○○○○○○○○○○○○○○○○○○○○○○○○○○○
    }
}

\section{はじめに}
修士中間試問会の発表者は資料を作成し、指導教員、アドバイザおよび教務補佐まで提出することになっています。

資料は、修士論文のExtended abstractです。指導教員および2名のアドバイザに事前に送付して読んでいただきます。評価シートの評価はこの資料および試問会における発表に基づいて行われます。また、資料は専攻内で電子的に閲覧可能な状態とし、試問会当日には会場に回覧資料として配置します。\renewcommand{\thefootnote}{\fnsymbol{footnote}}
\footnote[0]{
    {\normalsize
    \noindent{所属: }\\
    \noindent{指導教員: } \\
    }
}
\renewcommand{\thefootnote}{\arabic{footnote}}

\section{原稿作成の概要}
\subsection{表題}
表題は研究内容を適切に表すものにすること。「修士論文中間試問会原稿」などは不可。

\subsection{構成}
修士論文のExtended abstractとして、研究の背景、目的、方法、現在までに得られている結果とそれに関する考察、修士論文作成までの予定などを記述すること。冒頭には要旨を記載すること。

\subsection{使用言語}
日本語または英語

\subsection{ページ数}
8ページ以上

\subsection{用紙サイズと段組・レイアウト}
A42段組 (レターサイズや1段組は不可)

\qquad \qquad ただし、要旨は1段組とすること。

マージン\\
\qquad \qquad 上マージン30 mm\\
\qquad \qquad 下マージン27 mm\\
\qquad \qquad 左マージン18 mm\\
\qquad \qquad 右マージン18 mm\\
\qquad \qquad コラム間はマージン7 mmを目安とする

表題、著者名、所属の配置は見本に従うこと。

ページ番号はつけない。

\subsection{フォント}
英文:Times, Times New Romanなど

和文:MS 明朝・ヒラギノ明朝など明朝フォントを使用すること。

\subsection{文字サイズ(厳守)と行間}
文字サイズ\\
\qquad \qquad 表題12ポイント\\
\qquad \qquad 著者名10.5ポイント\\
\qquad \qquad 本文見出し 10.5 ポイント\\
\qquad \qquad 本文 10.5 ポイント\\

行間\\
\qquad \qquad シングルスペース(行間1行)

\subsection{所属の表記方法}
所属: [講座名]・[分野名] 

指導教員: [指導教員名] \qquad とする。\\

例: 社会情報モデル講座・分散情報システム分野\\
\qquad ~ 指導教員: 吉川正俊

1ページ目左下に記載すること。

\subsection{写真や画像の解像度}
PDF を作成する際に写真や画像の解像度が低くなりすぎないよう注意すること。例えば Acrobat Distiller を利用する場合、標準の設定では画像の解像度が低くなりすぎる場合があるので注意すること。300 DPI 以上の解像度を目安にすること。

\subsection{カラー画像などについて}
カラーの使用に制限はありません。ただし、試問会当日に会場で回覧される資料はモノクロで作成されます。

\subsection{ファイルの形式}
原稿は任意のツールで作成してよいが、最終的にはPDFファイルとして提出すること。

\section{資料提出の概要}
\subsection{提出方法}
原稿を作成要領に従って作成した後、PDFファイルを作成し、京都大学学習支援サービスPandA(https://panda.ecs.kyoto-u.ac.jp/portal)を通じて提出して下さい。PandAにログイン後、「2022社会情報学専攻」の課題として添付してください。ファイルサイズは10MBを超えないようにして下さい。

\subsection{ファイル名の付け方}
提出するファイルのファイル名は以下のようにして下さい。

\qquad \qquad <LastName>\_<FirstName>.pdf

\qquad \qquad \quad 例)Shakai\_Ichiro.pdf

\end{document}

